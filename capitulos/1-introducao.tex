\chapter{Introdução} \label{cap:introducao}

A medição é um processo auxiliar essencial para o desenvolvimento de software
com qualidade. É importante medir para entender e controlar processos, produtos 
e projetos \cite{ministerio_processo2012}. Medidas fornecem informações sobre 
objetos (processos, produtos e projetos) e eventos, tornando-os compreensíveis 
e controláveis \cite{fenton&pfleenger98}. Entretanto, muitas vezes, mede-se
simplesmente por medir \cite{ministerio_processo2012}, não justificando todo o 
esforço gasto durante o processo.

O cenário apresentado acima é o que geralmente ocorre quando as pessoas não estão
acostumadas a trabalhar no dado contexto. Quando as pessoas não sabem trabalhar
com determinadas métricas de software existem duas opções: utilizam de maneira
inadequada, não agregando valor ao processo de desenvolvimento de software; ou
apenas não utilizam, mesmo que sejam importantes para o processo de
desenvolvimento.

Muitas vezes, as métricas de código-fonte se encaixam bem nessa situação, não
são utilizadas dentro do processo de desenvolvimento porque não se sabe o que
deve ser feito com as mesmas e que decisões tomar. Quando utilizadas, são
utilizadas apenas as métricas de \textit{design} de código-fonte mais
conhecidas, como as métricas orientadas a objetos de
\citeonline{chidamber&kemerer2002}, isso ocorre porque são métricas já
consolidadas e possuem várias diretrizes de como utilizá-las de maneira
adequada. 

Neste trabalho, nós argumentamos que outras classes de métricas de código-fonte,
como as métricas de ameaças de vulnerabilidade, por exemplo, não são tão
utilizadas no desenvolvimento de software, devido a falta de conhecimento de
como interpretá-las e monitorá-las. Nesse contexto ao investigar como monitorar
métricas de ameaças de vulnerabilidade de código-fonte, de modo que ela possa
ser utilizada e interpretada de maneira que agregue valor ao produto/negócio,
estamos propondo modelos preditivos para essa classe de métrica.

\section{Justificativa}

Sabendo que existe o problema dentro da Engenharia de Software de interpretação
e monitoramento de métricas de código-fonte além das mais conhecidas, que são 
as métricas de \textit{design}, espera-se conseguir auxiliar os 
Engenheiros de Software com este trabalho. Tentando apresentar uma forma de
acompanhar e monitorar essas diferentes métricas que estão emergindo
nos últimos tempos. Muitos não as utilizam devido a falta de conhecimento sobre
o que fazer com essas informações advindas dessas classes de métricas.

Relacionado às métricas de ameaças de vulnerabilidade de código-fonte, existe a
questão da segurança dos softwares, que está em voga nos últimos tempos,
o que vem aumentando cada vez mais a busca por requisitos não funcionais relacionados a
segurança do software desenvolvido. Isso reforça a necessidade de inserção de
métricas de ameaças de vulnerabilidade de código-fonte dentro do ciclo de
desenvolvimento de software, a fim de quantificar alguns aspectos do software
até então não mensurados. Como foi mencionado anteriormente, não se consegue
controlar um aspecto do desenvolvimento do software enquanto não se pode
medi-lo. A identificação dessas ameaças de vulnerabilidades ainda dentro do ciclo de
desenvolvimento facilita a correção das mesmas, além de evitar custos de
possíveis futuras manutenções corretivas. Mas para isso, é necessário saber como
monitorá-las e acompanhá-las no decorrer do ciclo de desenvolvimento.

Existem algumas iniciativas de pesquisas relacionadas a como associar
características de \textit{design} do código-fonte com vulnerabilidades, como
nos trabalhos de \citeonline{arthur&carlos2014} e
\citeonline{alshammari2009}. Entretanto, isso não se mostra suficiente para a
inserção dessas métricas no ciclo de desenvolvimento de software. Além disso, é
necessário encontrar uma forma mais objetiva para auxiliar Engenheiros de
Software no monitoramento das métricas de ameaças de vulnerabilidade de código
fonte, sendo esse um dos possíveis motivos para a sua não utilização.

Segundo um artigo divulgado pelo \textit{NIST} (\textit{National Institute of
Standards and Technology}), escrito por \citeonline{jansen2009}, a coleta
de dados de projetos existentes e a análise dos mesmos podem indicar padrões e
informações importantes para a medição da segurança de software, sendo essa
apontada como uma possível área de pesquisa. Isso corrobora com a ideia deste
trabalho, onde foram definidos modelos preditivos de referência para algumas
métricas de ameaças de vulnerabilidade de código-fonte baseado em um software de
referência, analisando temporalmente as suas versões.


\section{Objetivos} \label{sec:objetivos}

Levando em consideração o contexto apresentado anteriormente, o objetivo deste
trabalho é encontrar um modelo preditivo para métricas de ameaças de
vulnerabilidade de código-fonte que sirva de referência para outros projetos de
software.

Para isso, outros objetivos paralelos precisam ser atingidos, como o
entendimento do processo de análise estática, principalmente o que diz respeito
a segurança de código-fonte, como as ferramentas que realizam essa análise
funcionam, além de algum processo de análise estatística que auxilie no
desenvolvimento da pesquisa.

Em resumo, o principal objetivo de pesquisa deste trabalho é responder a
seguinte questão-problema:

\begin{center}
  \textit{É possível o desenvolvimento de modelos preditivos de referência, para
  acompanhamento e monitoramente de métricas de ameaças de vulnerabilidade de
código-fonte em projetos de software?}
\end{center}

Para responder essa questão-problema foram levantadas as seguintes hipóteses
a serem respondidas:

\begin{itemize}
  \item \textit{H1}: As métricas de ameaças de vulnerabilidade de código-fonte
  podem ser observadas de maneira similar às métricas de \textit{design} de código
  fonte.

  \item \textit{H2}: Os valores das métricas de ameaças de vulnerabilidade de
    código-fonte se comportam como distribuições estatísticas de cauda longa, e
    não distribuições estatísticas normalizáveis, assim como acontece com as
    métricas de \textit{design}.

  \item \textit{H3}: A definição de um modelo baseado em uma função polinomial
   simples possibilita o monitoramento e acompanhamento das métricas de ameaças
   de vulnerabilidade de código-fonte.
\end{itemize}


\section{Organização do Trabalho}

Este trabalho está dividido em seis capítulos subsequentes. No capítulo
\ref{chap:analiseestatica}, são apresentados alguns conceitos como o de análise
estática e métricas de software, além de apresentar algumas peculiaridades sobre
a classificação e taxonomia de vulnerabilidades de software para em seguida
serem apresentadas as métricas de ameaças de vulnerabilidade de código-fonte
utilizadas neste trabalho. No capítulo \ref{chap:ferramentas}, é apresentada a
importância de uma ferramenta que automatize o processo de análise estática,
como essas ferramentas funcionam em geral e em seguida é aprofundado sobre cada
uma das etapas realizadas pelas ferramentas de análise estática de segurança de
código-fonte. No capítulo \ref{eda}, é apresentado a diferença da análise
exploratória de dados para as abordagens estatísticas tradicionais, em seguida é
detalhado cada uma das etapas realizadas durante esse processo. No capítulo
\ref{metodologia}, contém toda a metodologia desenvolvida nesta pesquisa,
seguindo uma abordagem de análise exploratória de dados, para poder se definir
modelos preditivos para as métricas em questão. No capítulo
\ref{definicaomodelos}, são apresentados as definições e validações dos modelos
preditivos das métricas de ameaças de vulnerabilidade de código-fonte,
continuando a abordagem estatística apresentada anteriormente, chegando ao final
com fórmulas matemáticas que representem-os. No capítulo \ref{chap:conclusao},
sendo esse o último capítulo do trabalho, serão feitas algumas conclusões,
respondendo a questão de pesquisa levantada, além de apresentar algumas
limitações e riscos que podem ameaçar este trabalho e uma lista de trabalhos
futuros.


