\chapter{Introdução} \label{cap:introducao}

Introdução

\section{Justificativa}

\section{Objetivos} \label{sec:objetivos}

\section{Metodologia}

\section{Organização do Trabalho}

Este trabalho está dividido em mais três (3) capítulos subsequentes. No capítulo
\ref{chap:metricas}, são apresentados conceitos teóricos sobre métricas de
software, passando por métricas de \textit{design} de código fonte até métricas de 
vulnerabilidade de código fonte. Nesse capítulo será possível entender um pouco mais 
sobre o objetivo das métricas de software e o porque da importância de se monitorar
métricas de vulnerabilidade. No capítulo \ref{estudodecaso}, é apresentado o
estudo de caso inicial realizado para este trabalho. Nele conterá a metodologia
utilizada, englobando hipóteses iniciais, as ferramentas utilizadas e o teste de
algumas hipóteses, além de uma análise qualitativa das métricas de
vulnerabilidade de código fonte. E por último, no capítulo
\ref{chap:consideracoes}, são feitas algumas considerações acerca do trabalho
feito até o momento, além de atividades a serem realizadas dentro do contexto de
um cronograma de trabalho.

