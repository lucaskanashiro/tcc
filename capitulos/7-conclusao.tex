\chapter{Conclusão} \label{chap:conclusao}

Como pode-se observar nas análises realizadas sobre valores de métricas de
vulnerabilidade de código fonte, uma análise quantitativa não seria relevante
para o momento e sim uma análise qualitativa, a fim de entender o comportamento
e natureza dessas métricas. Na tentativa de realização de uma análise
quantitativa dentro do contexto do estudo de caso, foram negadas algumas
hipóteses e outra hipótese preferiu-se não testa-la. Dentre as hipóteses
apresentadas na subseção \ref{subsec:hipoteses}, as hipóteses \emph{H1} e \emph{H2}
foram negadas e a \emph{H3} não foi testada, as justificativas do porque dessas
decisões está na seção \ref{subsec:teste_hipotese}, onde foi descrito o teste
das hipóteses no estudo de caso realizado.

Na realização da análise qualitativa dos valores das métricas de vulnerabilidade
de código fonte foi possível identificar métricas de vulnerabilidade mais
recorrentes, sendo algumas delas: DNP, AN, ROGU e AUV. Também foi possível
levantar uma nova hipótese a ser testada, onde em projetos de software com
propósitos similares podem também acompanhar suas métricas de vulnerabilidade de
código fonte de maneira semelhante.

A pesquisa limitou-se a uma revisão bibliográfica, um estudo de caso com análise 
quantitativa de um projeto de software e uma análise qualitativa de dez
projetos. Para a consolidação das análises espera-se aumentar a quantidade de
projetos analisados, a fim de ter um resultado mais consistente.

Possíveis limitações podem ser encontradas com relação a extração das métricas
de vulnerabilidade de código fonte, tais limitações devido a possíveis falsos
positivos e negativos, entretanto, seriam limitações da ferramenta \emph{Clang}.
Lembrando que a extração e coleta das métricas de vulnerabilidade não são foco
deste trabalho, apenas a forma de interpretação e acompanhamento das mesmas.

Tendo em vista o tema deste trabalho, será adicionado ao mesmo um novo tipo de
métrica de código fonte, sendo elas as
métricas de anotação para softwares implementados com a linguagem de programação
Java. Essas métricas introduzem um novo conceito de métricas de código fonte,
específicas para linguagem Java. O objetivo será o mesmo das métricas de
vulnerabilidade de código fonte, tentar determinar uma melhor forma para
interpretar e acompanhar métricas advindas dessa origem.

Após o entendimento das métricas estudadas, espera-se 
modelar o problema seguindo uma abordagem de aprendizado de
máquina, com o intuito de conseguir encontrar padrões existentes.

