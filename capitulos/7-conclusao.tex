\chapter{Conclusão} \label{chap:conclusao}

Na primeira etapa deste trabalho foi realizado um estudo qualitativo das
métricas de ameaças de vulnerabilidade de código-fonte, onde foram analisados um
total de dez projetos de software livre, a fim de entender o comportamento das
métricas em questão. Ao final, pode-se definir um subconjunto de métricas mais
frequentes e responder algumas das hipóteses apresentadas na Seção
\ref{sec:objetivos}. Na realização desse estudo foi possível identificar
métricas de ameaças de vulnerabilidade mais recorrentes, sendo elas: referência
a ponteiros nulos, variáveis não inicializadas e vazamento de memória, um resumo
das métricas calculadas de cada um dos projetos pode ser visto no Apêndice
\ref{anex:analise_qualitativa}. Além disso, foi respondida a hipótese levantada
na primeira parte deste trabalho, sendo essa a hipótese \textit{H1}. A hipótese
\textit{H1} afirma que as métricas de ameaças de vulnerabilidade de código-fonte
podem ser acompanhadas da mesma forma que as métricas de \textit{design}.
Após uma análise dos dados das métricas obtidos pode-se perceber que boa parte
dos valores das métricas de ameaças de vulnerabilidades são nulos, como pode-se
ver no Apêndice \ref{anex:percentis}, diferente das métricas de \textit{design},
o que altera a forma de acompanhamento das mesmas, isso é explicado até pela
natureza distinta das diferentes classes de métricas. Logo, pode-se negar a
hipótese \textit{H1}.

Na segunda etapa deste trabalho foi feito um trabalho estatístico para
determinar modelos de predição para as métricas de ameaças de vulnerabilidade de
código-fonte mais recorrentes. Para isso, foi usada uma abordagem de análise
exploratória de dados, onde nela se definiu o conjunto de dados referentes as
métricas que foram trabalhados, nesta etapa foi decidido a não definição de um
modelo de predição para a métrica relacionada a vazamento de memória (CWE401)
como foi explicado na Seção \ref{metodologia:eda}. Após a análise exploratória
dos dados ter sido realizada não se pode negar a hipótese \textit{H2}, que
afirma que os valores das métricas de ameaças de vunerabilidade se comportam
como uma distribuição estatística de cauda longa, não sendo similar a uma
distribuição normal, a análise realizada pode ser vista na Seção
\ref{metodologia:eda}. Com as informações obtidas através da análise pode-se
realizar a definição de modelos de predição para as métricas de ameaças de
vulnerabilidade de código-fonte relacionadas a referência de ponteiros nulos e
variáveis não inicializadas (CWE476 e CWE457), no Capítulo \ref{definicaomodelos}
pode-se ver o processo de definição, validação e seleção dos modelos
desenvolvidos, que apresentaram resultados satisfatórios apesar de possuírem uma
baixa complexidade (polinômio cúbico), como pode-se ver nos exemplos de uso
apresentados na Seção \ref{exemplosdeuso}. E com isso também não foi possível
negar a hipótese \textit{H3}, onde foi afirmado que pode-se monitorar métricas
de ameaças de vulnerabilidade através de um modelo baseado em uma função
polinomial.

Tendo sido analisadas as três hipóteses inicialmente levantadas para responder a
questão-problema deste trabalho, sendo ela: 

\begin{center}
  \textit{É possível o desenvolvimento de modelos preditivos de referência para
  acompanhamento e monitoramente de métricas de ameaças de vulnerabilidade de
código-fonte em projetos de software?}
\end{center}

Conclui-se que o desenvolvimento de modelos preditivos de referência, de baixa
complexidade, para métricas de ameaças de vulnerabilidade de código-fonte é
possível quando se possui um projeto de referência que possibilite a obtenção de
uma base de dados considerável referente a valores das mesmas. Como foi
discutido na Seção \ref{consolidacaoresultados}, quanto mais complexo for o
polinômio referente ao modelo, melhor o mesmo se adaptará aos dados reduzindo o
erro dentro do grupo de treinamento, entretanto, como o objetivo dos modelos
definidos neste trabalho é predizer esses valores para outros projetos de
software, o mesmo deve ter uma certa flexibilidade, evitando o
\textit{overfitting}. Além disso, modelos menos complexos tendem a gastar uma
menor quantidade de tempo para realizarem as suas predições, o que pode auxiliar a
inserção do mesmo no ciclo de desenvolvimento de software, tornando o processo
de obtenção de valores de referência dessas métricas mais rápido, facilitando o
monitoramento das mesmas em projetos de software.


\section{Limitações do Trabalho}

Uma possível limitação deste trabalho foi a identificação de \textit{outliers}
dentro do conjunto de dados das métricas trabalhadas, apresentada na Capítulo
\ref{definicaomodelos}. Nesse caso, um método simples e conhecido foi puramente
aplicado para a identificação de \textit{outliers}, não sendo encontrada nenhuma
evidência de que os valores realmente representavam algum evento fora do normal.

A principal limitação deste trabalho foi a não validação dos valores de
referência obtidos através dos modelos preditivos desenvolvidos com os reais
valores das métricas em diferentes projetos de software. Na Seção
\ref{exemplosdeuso} foram apresentados alguns valores de referência para alguns
projetos de software livre, entretanto, os mesmos não foram validados com os
reais valores das métricas que poderiam ser extraídos através de uma ferramenta
de análise estática de segurança de código. Esse seria um bom teste para os
modelos desenvolvidos.

\section{Trabalhos Futuros}

Como trabalho futuro deve-se validar os valores de referência preditos pelos
modelos desenvolvidos com os valores das métricas extraídas através de análise
estática e comparar o desempenho dos modelos. Para auxiliar no acompanhamento
das métricas de ameaças de vulnerabilidades de código-fonte em projetos menores,
com menor número de módulos, é interessante replicar este trabalho tomando como
referência um projeto desse porte.

Este trabalho também pode ser expandido a fim de definir modelos de predição
para outras métricas de ameaças de vulnerabilidade de código-fonte, aumentando o
abrangência desses modelos e facilitando a inserção dessa classe de métricas no
ciclo de desenvolvimento de software. Outro trabalho interessante seria
desenvolver modelos de predição de complexidade maior e comparar o seu
desempenho com os modelos de baixa complexidade desenvolvidos neste trabalho, e
verificar o custo benefício dos mesmos, podendo continuar a abordagem
apresentada na Seção \ref{evolucao_modelos}.

Outro trabalho interessante seria tentar definir modelos de predição de métricas de
ameaças de vulnerabilidade baseado em métricas de \textit{design}, ou seja, as
entradas do modelo seriam métricas de \textit{design} e como saída teriamos
valores de referência para métricas de ameaças de vulnerabilidade. Isso pode ser
vislumbrado já que, como foi apresentado no início deste trabalho, existem
trabalhos que tentam relacionar essas diferentes classes de métricas.
