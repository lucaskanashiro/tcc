\chapter{Metodologia}

Pretende-se nesse trabalho fazer uma análise qualitativa, utilizando distribuições estatísticas para determinar qual melhor
se associa a cada métrica de vulnerabilidade e possivelmente passando por percentis dos valores das mesmas. Para isso, serão
selecionados projetos de software livre para a realização de um estudo de caso especificado na seção \ref{estudodecaso}.
Serão coletadas métricas de vulnerabilidade de código fonte e as mesmas passarão por um processo de análise estatística
automatizado. 

De posse desses dados também espera-se determinar valores frequentes para as referidas métricas de vulnerabilidades de acordo
com a classe de sua aplicação.

Após a geração de todas essas informações será feita uma síntese e conclusão de como deve se dar o monitoramento de métricas de 
vulnerabilidade de código fonte dentro de projetos de software, assim podendo apontar trabalhos futuros.
