\chapter{Análise Estática de Código-fonte} \label{chap:analiseestatica}

A análise estática examina o código-fonte do programa e as razões sobre todos os
comportamentos possíveis que possam surgir em tempo de execução
\cite{ernst:2005}. Ou seja, é uma análise realizada sem a necessidade de
execução do código-fonte desenvolvido, não necessitando de suas dependências e
outras peculiaridades para realizá-la. Além disso, análise estática pode nos apresentar
características inerentes ao código-fonte, podendo as mesmas serem relacionadas
ao \textit{design} do código, apresentando-nos informações como complexidade dos
métodos e tamanho por exemplo.

Em geral, ao final de um processo de análise estática se tem alguma métrica de
software, sendo métrica de software uma função cujas entradas são os dados de
software e cuja saída é um único valor numérico, que pode ser interpretada como
o grau em que o software possui um determinado atributo que afeta sua qualidade
\cite{ieee:1061}. Entretanto, também existem análises estáticas que tem como
objetivo a identificação de não conformidades com o estilo do código por exemplo,
que podem não ter uma métrica de software como saída do processo.

Usualmente, análise estática de código-fonte remete à uma análise do
\textit{design} do código, da complexidade ou tamanho do mesmo, entretanto, essa
é uma visão que vem mudando devido aos requisitos não funcionais relacionados à
segurança do software ganharem cada vez mais importância em virtude do contexto
atual. Onde cada vez mais são desenvolvidas pesquisas acerca da análise estática
voltada para a segurança do código, tentando tornar esse tipo de requisito
quantificável e facilitar a inserção dessa classe de métricas no ciclo de
desenvolvimento de software.

Vários estudos vêm sendo realizados acerca da análise de código-fonte e
definição de métricas, além da sua relação com ameaças ou falhas em programas.
Sendo que um conjunto de ameças de vulnerabilidade pode ou não se tornar uma
vulnerabilidade dependendo do estado do sistema e abrir brechas para atacantes.
Como por exemplo em \citeonline{ferzund:2009,misra:2003,nagappan:2006,arthur&carlos2014}.
Isso demostra que cada vez mais existe a preocupação em volta da segurança e
ameaças de vulnerabilidade de software, e pensando em desenvolver e manter
software de uma maneira mais segura, necessita-se um melhor entendimento das
ameaças de vulnerabilidade e as vulnerabilidades conhecidas. Lembrando que a
análise estática de código-fonte é um forte aliado para garantir a segurança de
software, permitindo que equipes de desenvolvimento encontrem possíveis
vulnerabilidades durante o ciclo de desenvolvimento, reduzindo o custo de
futuras manutenções corretivas.

Serão apresentadas na Seção \ref{cwe} a taxonomia e classificação das
vulnerabilidades já conhecidas, e na Seção \ref{metricasvulnerabilidade} serão
explicadas algumas das métricas de ameaças de vulnerabilidade mais comuns em
projetos de software, sendo apresentadas as selecionadas para a realização deste
trabalho.

\section{Classificação e Taxonomia das Vulnerabilidade de Software}\label{cwe}

Segundo \citeonline{seacord&householder2005}, compreender as vulnerabilidades é
crucial para entender as ameaças que elas representam. Visando isso, alguns
trabalhos foram desenvolvidos acerca da classsificação das vulnerabilidades
existentes, tentou-se definir uma taxonomia para as mesmas. Taxonomia é o
processo científico de categorizar entidades, ou seja, organizá-las em grupos
\cite{gregio:2005}. Dividindo as vulnerabilidades em grupos passa a ser mais
fácil o entendimento de cada um delas e seus conteúdos. As características dos
grupos, chamadas características taxonômicas ou atributos, devem satisfazer às
seguinte propriedades segundo \citeonline{krsul:1998}: 

\begin{itemize}
  \item \textbf{Objetividade}: A característica deve ser identificada a partir
    do conhecimento objetivo e não do conhecimento subjetivo. O atributo
    mensurado deve ser claramente observado.

  \item \textbf{Determinismo}: Deve haver um processo claro que pode ser seguido
    para se extrair a característica.

  \item \textbf{Repetibilidade}: Várias pessoas, independentemente, extraindo a
    mesma característica do objeto devem concordar com o valor observado.

  \item \textbf{Mutuamente exclusiva}: A categorização em um grupo exclui a
    categorização em qualquer outro grupo.

  \item \textbf{Exaustiva}: Juntos, os grupos incluem todas as possibilidades.

  \item \textbf{Aceitável}: Lógica e intuitiva, de forma que as categorias
    possam ser aceitas pela comunidade.

  \item \textbf{Útil}: Pode ser utilizada para a obtenção de conhecimento no
    campo de pesquisa.
\end{itemize}

Portanto, se as propriedades apresentadas não forem satisfeitas não existe
taxonomia, sendo apenas uma simples classificação de vulnerabilidades. Vários
trabalhos foram desenvolvido com o intuito de desenvolver uma taxonomia para as
vulnerabilidades de software, como em
\citeonline{krsul:1998,aslam:1996,landwehr:1994}. A primeira proposta de
taxonomia para vulnerabilidades de software foi proposta em
\citeonline{abbott:1976}, esse estudo ficou conhecido como \textit{RISOS}
(\textit{Research Into Secure Operating Systems}), que tinha como intuito
auxiliar administradores de sistemas a entender aspectos de segurança dos
sistemas operacionais para prover uma melhor segurança para os mesmos. Nesse
caso, as vulnerabilidades foram agrupadas em sete classes, sendo elas:

\begin{enumerate}
  \item Validação incompleta dos parâmetros
  \item Validação inconsistente dos parâmetros
  \item Compartilhamento implícito de privilégios ou dados confidenciais
  \item Validação assíncrona ou serialização inadequada
  \item Autorização ou autenticação ou identificação inadequadas
  \item Violação de proibição ou limite
  \item Erro explorável de lógica
\end{enumerate}

Vários outros estudos foram derivados a partir do \textit{RISOS}, entretanto, o
que acontece é que essas taxonomias estão em desuso e a tendência atual é a
adoção de esquemas de classificação de vulnerabilidades, ou seja, o que acontece
na prática não repeita as propriedades de uma taxonomia.

Apesar da classificação ser o método mais utilizado nos dias de hoje, ainda não
existe um consenso de uma forma correta de se classificar vulnerabilidades. Uma
das mais conhecidas é a CVE (\textit{Common Vulnerabilities and Exposures}), que
é mantida pelo \textit{MITRE}\footnote{\url{http://www.mitre.org/}}, sendo essa
uma lista de nomes padronizados para vulnerabilidades e fraquezas de sistemas e
softwares, que tem por objetivo padronizar as vulnerabilidades e fraquezas já
conhecidas, facilitando o compartilhamento de informações encontradas de ano a
ano \cite{gregio:2005}. Esse é um trabalho de extrema importância vide que
algumas organizações nomeavam a mesma vulnerabilidade com nomes diferentes no
passado, dificultando o entendimento das mesmas.

Mesmo com o uso das CVEs, viu-se a necessidade das empresas e organizações de
utilizarem uma terminologia padrão para listar e classificar vulnerabilidades de
software, gerando uma base de dados unificada e uma base para ferramentas e
serviços de medição dessas vulnerabilidades, sendo assim foram desenvolvidas as
CWEs \cite{arthur&carlos2014}. A CWE\footnote{\url{https://cwe.mitre.org/}}
(\textit{Common Weakness Enumeration}) é uma lista formal de vulnerabilidades
comuns de software, tendo como objetivo estabelecer uma linguagem comum para
descrever ameaças de vulnerabilidade de software no \textit{design}, arquitetura
ou no código \cite{arthur&carlos2014}.  A principal diferença entre as CVEs e
CWEs é que as CVEs são vulnerabilidades que já ocorreram e são conhecidas pela
comunidade, já as CWEs estão relacionadas a fraquezas de software, apresentando
códigos e práticas de programação que podem se tornar futuras vulnerabilidades
do software.

As métricas apresentadas na Seção \ref{metricasvulnerabilidade} foram definidas
baseadas nas CWEs mais recorrentes em projetos de software livre, sendo esse um
indicativo que devem ser monitoradas em um projeto se software, conforme
discutido no Apêndice \ref{anex:analise_qualitativa}, referente a primeira etapa
deste trabalho.

\section{Métricas de Ameaças de Vulnerabilidade de Código-fonte}\label{metricasvulnerabilidade}

As métricas de ameaças de vulneabilidade de código-fonte apresentadas nesta
seção foram baseadas no estudo realizado na primeira parte desta pesquisa,
detalhes sobre essas decisões são discutidos no Capítulo \ref{metodologia}.

As principais ameaças de vulnerabilidade encontradas em projetos de software
livre foram: referência a ponteiros nulos, variáveis não inicializadas e
vazamento de memória. Sendo a
CWE476\footnote{\url{https://cwe.mitre.org/data/definitions/476.html}} referente
a referência a ponteiros nulos, a
CWE457\footnote{\url{https://cwe.mitre.org/data/definitions/457.html}} referente
a variáveis não inicializadas e a
CWE401\footnote{\url{https://cwe.mitre.org/data/definitions/401.html}} referente
ao vazamento de memória. A definição das métricas relacionadas a cada uma das
ameaças de vulneabilidade de código-fonte foi bem simples, sendo obtida a partir
do somatório da quantidade de ameaças encontradas no projeto dividido pelo
número total de módulos, dando uma taxa de cada uma das ameaças de
vulnerabilidade por módulo, podendo o valor da métrica variar entre 0.0 e 1.0.

Aprofundando mais em cada uma das ameaças de vulnerabilidade a fim de
compreendê-las, serão apresentadas uma a uma a seguir, nas Seções \ref{cwe476},
\ref{cwe457} e \ref{cwe401}.

\subsection{CWE476 - Referência a Ponteiros Nulos}\label{cwe476}

A ameaça de vulnerabilidade de referência a ponteiros nulos acontece quando
a aplicação tenta acessar o conteúdo da região de memória a que o ponteiro deveria estar
apontando, entretanto, o ponteiro está nulo, devido a algum estado inesperado do
programa. Tipicamente, esse tipo de ocorrência aborta a execução do programa.
Esse tipo de ameaça de vulnerabilidade pode ocorrer desde um pequeno erro de
programação, esquecendo de validar algum ponteiro antes de acessá-lo, até
complexas condições de corrida, onde existam dois processos paralelos que acessam o mesmo
ponteiro, e um libere a região de memória antes do outro finalizar a utilização
da mesma em um estado do programa específico. 

No Código \ref{codigoCWE476} pode-se ver um exemplo simples onde é chamada uma
função que deveria retornar um ponteiro para uma estrutura de dados e um campo
dessa estrutura é passado para uma chamada de sistema, entretanto, pode haver um
caso específico onde a função chamada não retorne o ponteiro esperado, e nesse
caso, faça referência a um ponteiro nulo.

\begin{lstlisting}[caption={Código exemplo CWE476}, label=codigoCWE476]
typedef struct command_ {
   char* program;
   char* input;
} command;

void main() {
    command *p = get_command();
    system(p->program, p->input);
} 
\end{lstlisting}

O exemplo do Código \ref{codigoCWE476} apresenta uma ameaça de vulnerabildade, pois o
conteúdo para onde o ponteiro aponta pode aparentemente sempre vir certo,
entretanto, em algum momento, com algumas entradas específicas o ponteiro
retornado pode ser nulo, caracterizando uma ameaça de vulnerabilidade de código
fonte. A eliminação da ameaça de vulnerabilidade é bastante simples nesse caso,
apenas é necesária a validação do ponteiro antes de sua chamada, bastando fazer
algo similar ao que é apresentado no Código \ref{fixcwe476}.

\begin{lstlisting}[caption={Evitando ameaça de vulnerabilidade CWE476},
label=fixcwe476]
typedef struct command_ {
   char* program;
   char* input;
} command;

void main() {
    command *p = get_command();

    if(p != NULL) {
        system(p->program, p->input);
    }
} 
\end{lstlisting}

Lembrando que essa ameaça de vulnerabilidade só pode ocorrer em linguagens de
programação que permitem a manipulação de ponteiros, dando liberdade para
gerenciar a memória utilizada pelo programa, sendo possível em linguagens tais
como \textit{C} e \textit{C++}.

Nesse caso específico apresentado ainda abre brechas para outras ameaças de
vulnerabilidade, onde um atacante poderia executar algum comando no sistema com
o usuário que está executando o processo do programa, já que o mesmo faz uma
chamada de sistema sem validar os seus parâmetros.

\subsection{CWE457 - Variáveis não Inicializadas}\label{cwe457}

A ameaça de vulnerabilidade de código-fonte relacionada a variáveis não
inicializadas ocorre quando o programa faz uso de uma variável que ainda não foi
inicializada pelo mesmo, tendo que lidar com certas coisas que pode levar o
pragrama para um estado imprevisível. Em algumas linguagens como \textit{C}, as
variáveis quando são declaradas não são inicializadas, e
quando isso aconte o conteúdo daquela variável passa a ser algo que estava
anteriormente naquela região de memória alocada, nesse caso, um atacante pode de
alguma forma conseguir escrever algo nessa região de memória inicializando a
variável com algum valor malicioso. Em outras linguagens onde as variáveis
quando declaradas são inicializadas com algum valor \textit{default}, dependendo
da implementação do programa, pode indicar um erro de tipo de variável em tempo
de execução, abortando o programa.

No Código \ref{codigocwe457} é exemplificado essa ameaça de vulnerabilidade de
código-fonte, onde se inicializa a variável apenas se o retorno de uma função é
verdadeiro. Nessa situação, quando o retorno for falso a variável não será
inicializada, entretanto, ainda será acessada, ocasionando um erro no programa.

\begin{lstlisting}[caption={Código exemplo CWE457}, label=codigocwe457]
void main() {
    int x;

    if(isTrue()) {
        x = 10;
    }

    manipulate(x);   
}
\end{lstlisting}

No melhor caso, se a variável não for inicializada o programa irá abortar,
entretanto, um atacante pode ter escrito algum código ou dado malicioso na
região de memória onde se localiza a pilha de variáveis do programa, podendo ter
inicializado a variável com algo malicioso, e essa variável será chamada
levando o programa a um estado imprevisível. Uma possível solução para esse
problema pode ser inicializar a variável com algum valor \textit{default} que a
variável não possa assumir, e ao chamar o método que manipula a variável
verificar se o conteúdo da variável ainda é o valor \textit{default}, como
pode-se ver no Código \ref{fixcwe457}.

\begin{lstlisting}[caption={Evitando ameaça de vulnerabilidade CWE457},
label=fixcwe457]
void main() {
    int x = -1;

    if(isTrue()) {
        x = 10;
    }

    if(x != -1) {
        manipulate(x);   
    }
}
\end{lstlisting}

\subsection{CWE401 - Vazamento de Memória}\label{cwe401}

A ameaça de vulnerabilidade de vazamento de memória ocorre quando o programa não
consegue gerenciar a liberação de memória após a sua utilização de maneira
aceitável, consumindo toda a memória alocada para o \textit{HEAP} do programa aos
poucos, até não restar mais. Esse tipo de erro acaba encerrando o programa em
algum momento inesperado, o que pode acabar gerando algum tipo de perda de dados
por exemplo.

No Código \ref{codigoCWE401} é exemplificado a ameaça de vulnerabilidade de
código-fonte, onde uma função aloca memória para as variáveis e quando a função
de liberação de memória é chamada, a mesma esquece de desalocar a região de
memória de uma das variáveis. Isso não seria um problema muito grande se não
houvesse várias requisições a essas funções, onde em algum ponto não restará
mais memória para o programa alocar novas variável e o mesmo será encerrado.

\begin{lstlisting}[caption={Código exemplo CWE401}, label=codigoCWE401]
foo connect() {
    foo foo = malloc(2048);
    foo bar = malloc(1024);

    return foo;
}

void destroy(foo) {
    free(foo);
}

void main()
{
    while(get_request_for_connection()) {
        foo var = connect();
        destroy(var);
    }
}
\end{lstlisting}

Esse pequeno erro de programação apresentado pode levar o programa a ser
encerrado no meio da sua execução, nesse caso a solução é bastante simple,
basta remover o segundo \textit{malloc()} da função \textit{connect()}. Porém,
podem haver casos mais críticos de vazamento de memória, onde possa haver algum
tipo de vulnerabilidade no programa que permita um atacante puxar algum gatilho
para que haja vazamento de memória, dessa forma podendo caracterizar um ataque
de negação de serviço (\textit{Denial of Service Attack}).

Esse tipo de ameaça de vulnerabilidade assim como a apresentada na Seção
\ref{cwe476}, só é possível em programas escritos em linguagens de programação
que dê liberdade para o seu próprio gerenciamento de memória, como as linguagens
\textit{C} e \textit{C++}.



Com base nas métricas de ameaças de vulnerabilidade de código-fonte aqui
apresentadas, foi realizada uma análise ao decorrer do tempo das versões do
projeto \textit{Linux Kernel}, a fim de definir modelos de predição de
referência para essas métricas que auxiliem a monitoramento das mesmas no ciclo
de desenvolvimento de software.


