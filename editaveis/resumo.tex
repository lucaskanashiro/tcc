\begin{resumo}

Devido a constante evolução da Engenharia de Software, a cada dia surgem 
novas linguagens de programação, paradigmas de desenvolvimento, formas de
avaliar processos, entre outras coisas. Com as métricas de código-fonte não 
é diferente, com o passar do tempo surgem outras classes de métricas e para
a utilização das mesmas vem a necessidade de se saber como utiliza-las. Para a
utilização de uma métrica de software qual for, é necesário ter conhecimento
sobre como realizar a coleta, cálculo, interpretação e análise para tomada de
decisões. No contexto das métricas de código-fonte, a coleta e cálculo na
maioria das vezes são automatizadas por ferramentas, mas como acompanhá-las e
monitorá-las de maneira correta no decorrer do ciclo de desenvolvimento de
software? Este trabalho visa auxiliar o Engenheiro de Software a monitorar e
acompanhar métricas de ameaças de vulnerabilidade de código-fonte através de um
modelo de predição de referência, tendo em vista que cada vez mais os softwares
possuem requisitos não funcionais de segurança, o que leva a necessidade de saber
como monitorar esses requisitos durante o ciclo de desenvolvimento de software.

 \vspace{\onelineskip}
    
 \noindent
 \textbf{Palavras-chaves}: Métricas; Vulnerabilidade; Código-fonte; Predição.
\end{resumo}
